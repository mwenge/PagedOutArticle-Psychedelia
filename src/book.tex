\documentclass[pagedoutpaper,twocolumn,9pt]{pagedout}
\pagenumbering{gobble}
\usepackage[utf8]{inputenc}
\usepackage[hidelinks]{hyperref}
\usepackage[table]{xcolor}

\definecolor{mGreen}{rgb}{0,0.6,0}
\definecolor{mGray}{rgb}{0.5,0.5,0.5}
\definecolor{keyword_color}{rgb}{0.00,0,1.0}
\definecolor{bg_color}{HTML}{fffFfb}

\usepackage{float}
\usepackage{adjustbox}
\usepackage{booktabs}  

\usepackage[justification=centering,labelfont=bf]{caption}
\captionsetup[subfigure]{font=footnotesize,labelfont=footnotesize,labelformat=empty}

\usepackage{subcaption}

\usepackage[final]{listings}
\lstset{ 
  backgroundcolor=\color{bg_color},
  breaklines=true,
   frame=single,
   basicstyle=\footnotesize\ttfamily,
   xleftmargin=9pt,
   framexleftmargin=5pt,
   showstringspaces=false,
   aboveskip=12pt,
   belowskip=0pt,
   % Make some parts appear in red.
   moredelim=**[is][\color{red}]{@}{@},
}
\lstdefinelanguage
   [6502]{Assembler}     % add a "x64" dialect of Assembler
   [x86masm]{Assembler} % based on the "x86masm" dialect
   % with these extra keywords:
   {morekeywords={.BYTE,LDY,LDX,INC,LDA,BEQ,BNE,JSR,DEY,DEX,PHA,PLA,INY, %
									STA,SEC,SBC,BCS,EOR,RTS,SEI,TXA,ASL,TAY,INX,CPY,TAX,RTI,}} % etc.


\lstdefinestyle{6502Style}{
    backgroundcolor=\color{bg_color},   
    commentstyle=\color{mGreen},
    keywordstyle=\color{keyword_color},
    numberstyle=\color{mGray},
    stringstyle=\color{mGreen},
    breakatwhitespace=false,         
    breaklines=true,                 
    captionpos=b,                    
    keepspaces=true,                                  
    showspaces=false,                
    showstringspaces=false,
    showtabs=false,                  
    tabsize=1,
    language=[6502]Assembler,
    morekeywords={ICALVE,CALVEC,RTSL,call, ld, or, jr,ldr, str,ldir, EI, IM, .area}
}
\lstloadlanguages{[6502]Assembler}

\newcommand*{\icode}[1]{{\texttt{#1}}}
\newcommand*{\wicode}[1]{{\texttt{\textcolor{white}{#1}}}}

\lstset{style=6502Style}
\lstset{ 
   aboveskip=5pt,
   belowskip=0pt,
}

\makeatletter
\newcommand{\fsize}{\f@size pt }
\newcommand{\textFontName}{\f@family}
\renewcommand{\maketitle}{
\begin{flushleft}
{\noindent\Huge\bf\@title}\break
\end{flushleft}
}
\makeatother


\usepackage{pgffor}
\usepackage{pgfplots}
\usepackage{makecell}
\usepackage{subfiles}
\usepackage{tikz}
\usetikzlibrary{matrix}
\usetikzlibrary{arrows.meta}
\newcommand*\circled[1]{\tikz[baseline=(char.base)]{\node[shape=circle,draw,inner sep=1pt] (char) {#1};}}
\usetikzlibrary{positioning,shapes}
%\usepgfplotslibrary{external} 
%\tikzexternalize


\definecolor{c64_black}{HTML}{000000}
\definecolor{c64_white}{HTML}{ffffff}
\definecolor{c64_red}{HTML}{880000}
\definecolor{c64_cyan}{HTML} {aaffee}
\definecolor{c64_violet}{HTML}{cc44cc}
\definecolor{c64_purple}{HTML}{cc44cc}
\definecolor{c64_green}{HTML}{00cc55}
\definecolor{c64_blue}{HTML} {0000aa}
\definecolor{c64_yellow}{HTML} {eeee77}
\definecolor{c64_orange}{HTML} {dd8855}
\definecolor{c64_brown}{HTML}{664400}
\definecolor{c64_lightred}{HTML}{ff7777}
\definecolor{c64_ltred}{HTML}{ff7777}
\definecolor{c64_darkgrey}{HTML}{333333}
\definecolor{c64_grey}{HTML} {333333}
\definecolor{c64_grey1}{HTML} {333333}
\definecolor{c64_grey2}{HTML} {777777}
\definecolor{c64_darkgray}{HTML}{333333}
\definecolor{c64_gray}{HTML} {333333}
\definecolor{c64_gray1}{HTML} {333333}
\definecolor{c64_gray2}{HTML} {777777}
\definecolor{c64_lightgreen}{HTML}{aaff66}
\definecolor{c64_lightblue}{HTML} {0088ff}
\definecolor{c64_lightgrey}{HTML}{bbbbbb}
\definecolor{c64_lightgray}{HTML}{bbbbbb}
\definecolor{c64_ltgreen}{HTML}{aaff66}
\definecolor{c64_ltblue}{HTML} {0088ff}
\definecolor{c64_ltgrey}{HTML}{bbbbbb}
\definecolor{c64_ltgray}{HTML}{bbbbbb}
\definecolor{c64_grey3}{HTML}{bbbbbb}
\definecolor{c64_gray3}{HTML}{bbbbbb}

\usepackage{fix-cm}    

\makeatletter
\newcommand\HUGE{\@setfontsize\Huge{32}{40}}
\makeatother   

\title{Psychedelia: A Puzzle}
\author{Rob Hogan}

\begin{document}    
\maketitle
\vspace{-0.8cm}
Rob Hogan, \href{https://psychedeliasyndro.me}{\textcolor{blue}{https://psychedeliasyndro.me}}
\begin{lstlisting}[basicstyle=\ttfamily\small]
starOneXPosOffsets        ;        5       
.BYTE 0,1,1,1,0,-1,-1,-1  ;                
.BYTE 0,2,0,-2            ;       4 4                                       
.BYTE 0,3,0,-3            ;        3                                        
.BYTE 0,4,0,-4            ;        2                                        
.BYTE -1,1,5,5,1,-1,-5,-5 ;        1       
.BYTE 0,7,0,-7            ;   4   000   4                                   
                          ; 5  3210 0123  5                  
starOneYPosOffsets        ;   4   000   4  
.BYTE -1,-1,0,1,1,1,0,-1  ;        1                                        
.BYTE -2,0,2,0            ;        2                                        
.BYTE -3,0,3,0            ;        3                                        
.BYTE -4,0,4,0            ;       4 4      
.BYTE -5,-5,-1,1,5,5,1,-1 ;                                                 
.BYTE -7,0,7,0            ;        5       
\end{lstlisting}   
This is an example of the data structure at the heart of Jeff Minter's \textit{Psychedelia}, the
first ever light synthesiser. It is the seed of the algorithm that Minter used in games such as
\textit{Tempest 2000} and the interactive music visualizer in the \textit{XBOX 360}. Maybe just
by looking at the code above you can guess that the values are X and Y offsets from a centre origin.
These build up the picture given in the comment section on the right. The numbers in that illustration
are an index into each line in the array: for example, the square of \icode{0}'s is from the first line in
both arrays. So given this information and assuming you have a colour table with the following values..
\begin{adjustbox}{width=5cm,center}
  \begin{tabular}{cccccccc}
    \cellcolor[HTML]{000000}\textcolor{white}{\icode{0}} & \cellcolor{c64_blue}\textcolor{white}{\icode{1}} &
    \cellcolor{c64_red}\textcolor{white}{\icode{2}} & \cellcolor{c64_purple}\textcolor{white}{\icode{3}} &
    \cellcolor{c64_green}\icode{4} & \cellcolor{c64_cyan}\icode{5} &
    \cellcolor{c64_yellow}\icode{6} & \cellcolor{c64_white}\icode{7}  \\
  \end{tabular}
\end{adjustbox}
..see if you can figure out the algorithm Minter used to generate the sequence below. I've enlarged the 
start of the sequence so you can begin to get a sense of how it operates. The numbers in the diagrams
represent the values in the colour table. The numbers in the color table and the number of iterations
made through the array each time are somehow related. See if you can figure it out.
If you're impatient to learn the answer, you can find it in the second and third chapters of 
\href{https://psychedeliasyndro.me}{\textcolor{blue}{https://psychedeliasyndro.me}}.

\begin{adjustbox}{width=8.5cm,center}
  \begin{tabular}{rl}
    \makecell[l]{
      \subfile{src/maps/pattern0_ram_stage0.tex}
      \hspace{0.1cm}
      \subfile{src/maps/pattern0_ram_stage1.tex}
      \hspace{0.1cm}
      \subfile{src/maps/pattern0_ram_stage2.tex}
    }\\
  \end{tabular}
\end{adjustbox}
\begin{adjustbox}{width=8.5cm,center}
  \begin{tabular}{rl}
    \makecell[l]{
      \subfile{src/maps/pattern0_ram_stage3.tex}
      \hspace{0.1cm}
      \subfile{src/maps/pattern0_ram_stage4.tex}
      \hspace{0.1cm}
      \subfile{src/maps/pattern0_ram_stage5.tex}
    }\\
  \end{tabular}
\end{adjustbox}
\begin{adjustbox}{width=8.5cm,center}
  \begin{tabular}{rl}
    \makecell[l]{
      \subfile{src/maps/pattern0_ram_stage6.tex}
      \hspace{0.1cm}
      \subfile{src/maps/pattern0_ram_stage7.tex}
      \hspace{0.1cm}
      \subfile{src/maps/pattern0_ram_stage8.tex}
    }\\
  \end{tabular}
\end{adjustbox}
\begin{adjustbox}{width=8.5cm,center}
  \begin{tabular}{rl}
    \makecell[l]{
      \subfile{src/maps/pattern0_ram_stage9.tex}
      \hspace{0.1cm}
      \subfile{src/maps/pattern0_ram_stage10.tex}
      \hspace{0.1cm}
      \subfile{src/maps/pattern0_ram_stage11.tex}
    }\\
  \end{tabular}
\end{adjustbox}
\begin{adjustbox}{width=8.5cm,center}
  \begin{tabular}{rl}
    \makecell[l]{
      \subfile{src/maps/pattern0_ram_stage12.tex}
      \hspace{0.1cm}
      \subfile{src/maps/pattern0_ram_stage13.tex}
      \hspace{0.1cm}
      \subfile{src/maps/pattern0_ram_stage14.tex}
      \hspace{0.1cm}
      \subfile{src/maps/pattern0_ram_stage15.tex}
    }\\
  \end{tabular}
\end{adjustbox}
\begin{adjustbox}{width=8.5cm,center}
  \begin{tabular}{rl}
    \makecell[l]{
      \subfile{src/maps/pattern0_ram_stage16.tex}
      \hspace{0.1cm}
      \subfile{src/maps/pattern0_ram_stage17.tex}
      \hspace{0.1cm}
      \subfile{src/maps/pattern0_ram_stage18.tex}
      \hspace{0.1cm}
      \subfile{src/maps/pattern0_ram_stage19.tex}
    }\\
  \end{tabular}
\end{adjustbox}
\begin{adjustbox}{width=8.5cm,center}
  \begin{tabular}{rl}
    \makecell[l]{
      \subfile{src/maps/pattern0_ram_stage20.tex}
      \hspace{0.1cm}
      \subfile{src/maps/pattern0_ram_stage21.tex}
      \hspace{0.1cm}
      \subfile{src/maps/pattern0_ram_stage22.tex}
      \hspace{0.1cm}
      \subfile{src/maps/pattern0_ram_stage23.tex}
    }\\
  \end{tabular}
\end{adjustbox}
\begin{adjustbox}{width=8.5cm,center}
  \begin{tabular}{rl}
    \makecell[l]{
      \subfile{src/maps/pattern0_ram_stage24.tex}
      \hspace{0.1cm}
      \subfile{src/maps/pattern0_ram_stage25.tex}
      \hspace{0.1cm}
      \subfile{src/maps/pattern0_ram_stage26.tex}
      \hspace{0.1cm}
      \subfile{src/maps/pattern0_ram_stage27.tex}
    }\\
  \end{tabular}
\end{adjustbox}
\begin{figure}[H]
    \centering
    \foreach \l in {28, 29, ..., 132}
    {
      \includegraphics[width=1.06cm]{src/maps/pixel_pattern_\l.png}%
    }%
\end{figure}
\end{document}

